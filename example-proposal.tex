% !TEX encoding = UTF-8 Unicode
% !TEX TS-program = xelatex
% !BIB program = biber

% MIT License
%
% Copyright (c) 2019 Star Brilliant
%
% Permission is hereby granted, free of charge, to any person obtaining a copy
% of this software and associated documentation files (the "Software"), to deal
% in the Software without restriction, including without limitation the rights
% to use, copy, modify, merge, publish, distribute, sublicense, and/or sell
% copies of the Software, and to permit persons to whom the Software is
% furnished to do so, subject to the following conditions:
%
% The above copyright notice and this permission notice shall be included in
% all copies or substantial portions of the Software.
%
% THE SOFTWARE IS PROVIDED "AS IS", WITHOUT WARRANTY OF ANY KIND, EXPRESS OR
% IMPLIED, INCLUDING BUT NOT LIMITED TO THE WARRANTIES OF MERCHANTABILITY,
% FITNESS FOR A PARTICULAR PURPOSE AND NONINFRINGEMENT. IN NO EVENT SHALL THE
% AUTHORS OR COPYRIGHT HOLDERS BE LIABLE FOR ANY CLAIM, DAMAGES OR OTHER
% LIABILITY, WHETHER IN AN ACTION OF CONTRACT, TORT OR OTHERWISE, ARISING FROM,
% OUT OF OR IN CONNECTION WITH THE SOFTWARE OR THE USE OR OTHER DEALINGS IN THE
% SOFTWARE.

\documentclass{HDU-Bachelor-Thesis-Proposal}

% 导入 \addbibresource \printbibliography
\usepackage{biblatex}
% 导入 \includepdf
\usepackage{pdfpages}
% 导入 \sout
\usepackage{ulem}
% 导入 \setmathfont
\usepackage{unicode-math}
% 导入 \textcolor
\usepackage{xcolor}

\title{杭州电子科技大学本科开题报告 \LaTeX{} 模版样例文档}
\author{张三}
\date{2021 年 3 月 19 日}
\HDUschool{计算机学院}
\HDUmajor{计算机科学与技术}
\HDUclassid{1234567890}
\HDUstudentid{1234567890}
\HDUadviser{李四}

% 从 ref.bib 中读入参考文献
\addbibresource{ref.bib}

% 设置公式字体(其它字体在模板中已设置好)
\setmathfont{texgyretermes-math.otf}

\begin{document}
% 页码在 PDF 阅读器中显示为罗马数字
\pagenumbering{roman}
% 关闭页眉、页脚
\pagestyle{empty}

% 生成封面
\maketitle
% 或者使用 Microsoft Word 制作封面后导入
%\includepdf[pages={-}]{封面.pdf}

\clearpage
% 将页码重置为 1
\pagenumbering{arabic}

\section{综述本课题国内外研究动态,说明选题的依据和意义}

\subsection{研究动态}

1999年11月11日,阿里巴巴高调发布人才招聘信息。当天阿里巴巴在《钱江晚报》第八版发布招聘广告,虽然这并不是第一次刊登招聘广告,却是第一次发出“If not now,when? If not me,who?”(此时此刻,非我莫属)的英雄帖,这句豪言壮语响当当地说出了“舍我其谁”的使命感和责任感,至今听来依旧热血沸腾,成为阿里人的经典土话\cite{ali-quotes},直至今天。

\subsection{研究意义}

“不是卖命,是一群有情有义的人在一起干一件有意义的事,为客户为这个社会贡献我们的一份力量”,这句话非常到位,外人难以体会到其中的良苦用心,即使许多阿里人也难以领会。于是,他们选择了泼脏水。你以为脏水泼向了谁?只是泼向了你自己的内心而已,如果你一直看向深渊,深渊也会看着你,希望大家都能心向光明,共勉。

希望价值观考核有一天真的可以取消,不是说到那一天价值观不存在了,而是已经渗透到每一个阿里人血脉之中,那是一种大音希声、大象无形、大道无言的愿景。

\section{研究的基本内容,拟解决的主要问题}

\subsection{基本内容}

复盘,赋能,抓手,对标,沉淀,对齐,拉通,倒逼,颗粒度,落地,中台,方法论,漏斗,组合拳,闭环,生命周期,打法,履约,引爆点,串联,价值转化,纽带,矩阵,协同,反哺,点线面,认知,强化认知,强化心智,交互,兼容,包装,响应,刺激,规模,重组,量化,宽松,资源倾斜,完善逻辑,抽离透传、抽象,复用打法,发力,精细化,布局,商业模式,联动,场景,聚焦,快速响应,影响力,价值,细分,垂直领域,维度,定性定量,聚焦,去中心化,关键路径,接地气,梳理,输出,格局,生态。

\subsection{主要问题}

你写这篇文章体现阿里哪条价值观?你的思维闭环是什么?你的抓手是什么?你的方法论是什么?帖子能容是否还能再更精进些?你写和别人写有什么区别?你写这篇文章产生的意义是什么?你的言论是否和你的P级相符合?你梳理的逻辑价值点又是什么?

\section{研究步骤、方法及措施}

\subsection{研究步骤}

\subsubsection{核心思想}

\paragraph{为过程鼓掌,为结果买单}

这要求我们做事要以结果为导向,要清楚我们的目标和方向,然后努力去达到那个目标。

\paragraph{合理的要求叫做锻炼,不合理的要求叫做磨炼}

假如上级很难搞,我就当成锻炼,反正今天不过这个关,明天还得过,周围的生活,社会的压力,这个东西大得多的还有,他骂就当锻炼,合理的要求叫做锻炼,不合理的要求叫做磨炼,我一直这样想,自己心里面很好的去释放这种所谓的压力,第二个在这个过程中,不断把我自己能够承受的压力再扩展、放大,今天其实没有什么压力过不去的,但是需要有时间安静。

\paragraph{你感觉不舒服的时候,就是成长的时候}

感觉不舒服的时候,人都希望通过某些途径和方法让自己舒服,于是不舒服成了是驱动人进步和成长的良药!有了不舒服,关键是要学会寻找合适的对症下药的方法。

\paragraph{你自己觉得有,别人感觉不到你有,你就是没有}

Review中的对话语。人要经常跳出来看自己,要从别人那里学会照镜子。并且要做到所思所言所行一致,则别人所见才将趋于一致。

\subsubsection{具体措施}

\paragraph{很傻很天真,又猛又持久}

对于梦想、目标和愿景我们傻傻地坚持,对于业绩我们迅速猛烈并且持久,“Double-digit growth every month”。

\paragraph{加班是应该的,不加班也是应该的,只有完不成工作是不应该的}

有同学提出“是否应该加班”的疑惑,马总给出了最好的答案。做好时间管理,积极改进工作方法,提高工作效率,完成工作,做好工作,至于是否加班,它只是一个表现。

\paragraph{今天最好的表现是明天最低的要求}

时常用这句话鼓励自己和团队,既是进取的表现,也是自信的表现。相信我们能做到,相信明天会更好!



\clearpage
\section{研究工作进度}

\begin{table}[h]%
	\centering
	\begin{tabular}{| c | c | c |}%
    \hline
	\bfseries 序号 & \bfseries 时间 & \bfseries 内容 \\ \hline
	\bfseries 1 & 2020.12.2--2020.12.25 & 选好毕业设计题目并准备相关资料 \\ \hline
	\bfseries 2 & 2020.12.26--2021.1.10 & 接受任务书 \\ \hline
    \bfseries 3 & 2021.1.11--2021.3.14 & 搜集资料,准备开题报告 \\ \hline
	\bfseries 4 & 2021.3.15--2021.3.19 & 开题报告会 \\ \hline
    \bfseries 5 & 2021.3.20--? & 找到抓手,思考毕设的方法论 \\ \hline
	\bfseries 6 & ?--? & 精细化业务颗粒度,重塑设计细节 \\ \hline
    \bfseries 7 & ?--? & 打出一套从编码到调试的组合拳 \\ \hline
	\bfseries 8 & ?--2021.5.20 & 复盘实际实现,赋能毕设论文 \\ \hline
    \bfseries 9 & 2021.5.21--2021.5.31 & 论文评审及查重 \\ \hline
	\bfseries 10 & 2021.6.1--2021.6.6 & 答辩报告会 \\ \hline
    \end{tabular}
\end{table}

% 列出参考文献
\section{主要参考文献}
% 修正目录超链接问题
\phantomsection{}
\printbibliography[heading=none]

% 生成小组评审意见表
\makeassessment
\end{document}
