% !TEX encoding = UTF-8 Unicode
% !TEX TS-program = xelatex
% !BIB program = biber

% MIT License
%
% Copyright (c) 2019 Star Brilliant
%
% Permission is hereby granted, free of charge, to any person obtaining a copy
% of this software and associated documentation files (the "Software"), to deal
% in the Software without restriction, including without limitation the rights
% to use, copy, modify, merge, publish, distribute, sublicense, and/or sell
% copies of the Software, and to permit persons to whom the Software is
% furnished to do so, subject to the following conditions:
%
% The above copyright notice and this permission notice shall be included in
% all copies or substantial portions of the Software.
%
% THE SOFTWARE IS PROVIDED "AS IS", WITHOUT WARRANTY OF ANY KIND, EXPRESS OR
% IMPLIED, INCLUDING BUT NOT LIMITED TO THE WARRANTIES OF MERCHANTABILITY,
% FITNESS FOR A PARTICULAR PURPOSE AND NONINFRINGEMENT. IN NO EVENT SHALL THE
% AUTHORS OR COPYRIGHT HOLDERS BE LIABLE FOR ANY CLAIM, DAMAGES OR OTHER
% LIABILITY, WHETHER IN AN ACTION OF CONTRACT, TORT OR OTHERWISE, ARISING FROM,
% OUT OF OR IN CONNECTION WITH THE SOFTWARE OR THE USE OR OTHER DEALINGS IN THE
% SOFTWARE.

\documentclass{HDU-Bachelor-Thesis-Proposal}

% 导入 \addbibresource \printbibliography
\usepackage{biblatex}
% 导入 \includepdf
\usepackage{pdfpages}
% 导入 \sout
\usepackage{ulem}
% 导入 \setmathfont
\usepackage{unicode-math}
% 导入 \textcolor
\usepackage{xcolor}

\title{杭州电子科技大学本科开题报告 \LaTeX{} 模版样例文档}
\author{张三}
\date{2021 年 3 月 19 日}
\HDUschool{计算机学院}
\HDUmajor{计算机科学与技术}
\HDUclassid{00000000}
\HDUstudentid{00000000}
\HDUadviser{李四}

% 从 ref.bib 中读入参考文献
\addbibresource{ref.bib}

% 设置公式字体(其它字体在模板中已设置好)
\setmathfont{texgyretermes-math.otf}

\begin{document}
% 页码在 PDF 阅读器中显示为罗马数字
\pagenumbering{roman}
% 关闭页眉、页脚
\pagestyle{empty}

% 生成封面
\maketitle
% 或者使用 Microsoft Word 制作封面后导入
%\includepdf[pages={-}]{封面.pdf}

\clearpage
% 将页码重置为 1
\pagenumbering{arabic}

\section{综述本课题国内外研究动态,说明选题的依据和意义}

\subsection{研究动态}

一般来说,解决论文选题的问题,首先要找到抓手。 所以,互联网产品经理间流传着这样一句话,此时此刻,非我莫属\cite{ali-quotes}。带着这句话,我们还要更加慎重的去对齐这个问题:所谓论文选题,关键是论文选题如何才能赋能目标,进而反哺目标生态。我认为,互联网产品经理间有着这样的共识,如何强化认知,快速响应,是非常值得跟进的。也许这句话就是最好的答案。所谓论文选题,关键是论文选题如何才能赋能目标,进而反哺目标生态。一般来说,互联网研发人员间流传着这样一句话,在细分领域找到抓手,形成方法论,才能对外输出,反哺生态。这不禁令我深思。经过上述讨论,经过上述讨论,既然如此,一般来说,我们认为,找到抓手,形成方法论,论文选题则会迎刃而解。带着这些问题,我们来聚焦一下——论文选题,经过上述讨论,而这些问题并不是痛点,而我们需要聚焦的关键是,我们都必须串联相关生态,去思考有关论文选题的问题。了解清楚论文选题到底存在于哪条赛道,是解决一切问题的痛点。

\subsection{研究意义}

总的来说,互联网研发人员间有着这样的共识,在细分领域找到抓手,形成方法论,才能对外输出,反哺生态。这句话语虽然很短,但沉淀之后真的能发现痛点。而这些问题并不是痛点,而我们需要聚焦的关键是,所谓论文选题,关键是论文选题如何才能赋能目标,进而反哺目标生态。

\section{研究的基本内容,拟解决的主要问题}

\subsection{基本内容}

互联网研发人员间流传着这样一句话,在细分领域找到抓手,形成方法论,才能对外输出,反哺生态。带着这句话,我们还要更加慎重的去对齐这个问题:互联网大厂管理人员间流传着这样一句话,找到正确的赛道,选择正确的商业模式,才能将项目落地。这句话语虽然很短,但沉淀之后真的能发现痛点。我们认为,找到抓手,形成方法论,毕设研究基本内容则会迎刃而解。

\subsection{主要问题}

互联网间有着这样的共识,好的产品要分析用户痛点,击穿用户心智。这让我明白了问题的抓手,一般来说,带着这些问题,我们来聚焦一下——毕设研究基本内容,所谓毕设研究基本内容,关键是毕设研究基本内容如何才能赋能目标,进而反哺目标生态。

\section{研究步骤、方法及措施}

\subsection{研究步骤}

\subsubsection{核心思想}

为过程鼓掌,为结果买单\cite{ali-quotes}。

\subsubsection{具体措施}

研究步骤、方法及措施的打法,到底是怎样的,而聚焦研究步骤、方法及措施本身,又会沉淀出什么样的方法论?互联网间流传着这样一句话,好的产品要分析用户痛点,击穿用户心智。这句话语虽然很短,但沉淀之后真的能发现痛点。互联网产品经理间流传着这样一句话,如何强化认知,快速响应,是非常值得跟进的。这让我明白了问题的抓手,总的来说,解决研究步骤、方法及措施的问题,首先要找到抓手。 所以,互联网大厂管理人员间有着这样的共识,找到正确的赛道,选择正确的商业模式,才能将项目落地。也许这句话就是最好的答案。总的来说,互联网产品经理间有着这样的共识,如何强化认知,快速响应,是非常值得跟进的。这不禁令我深思。互联网从业者间有着这样的共识,只有适度倾斜资源,才能赋能整体业务。这句话语虽然很短,但沉淀之后真的能发现痛点。互联网从业者间流传着这样一句话,只有适度倾斜资源,才能赋能整体业务。带着这句话,我们还要更加慎重的去对齐这个问题:互联网运营人员间有着这样的共识,做精细化运营,向目标发力,才能获得影响力。这让我明白了问题的抓手,我认为,互联网从业者间有着这样的共识,只有适度倾斜资源,才能赋能整体业务。也许这句话就是最好的答案。

\clearpage
\section{研究工作进度}

\begin{table}[h]%
	\centering
	\begin{tabular}{| c | c | c |}%
    \hline
	\bfseries 序号 & \bfseries 时间 & \bfseries 内容 \\ \hline
	\bfseries 1 & 2020.12.2--2020.12.25 & 选好毕业设计题目并准备相关资料 \\ \hline
	\bfseries 2 & 2020.12.26--2021.1.10 & 接受任务书 \\ \hline
    \bfseries 3 & 2021.1.11--2021.3.14 & 搜集资料,准备开题报告 \\ \hline
	\bfseries 4 & 2021.3.15--2021.3.19 & 开题报告会 \\ \hline
    \bfseries 5 & 2021.3.20--? & 找到抓手,思考毕设的方法论 \\ \hline
	\bfseries 6 & ?--? & 精细化业务颗粒度,重塑设计细节 \\ \hline
    \bfseries 7 & ?--? & 打出一套从编码到调试的组合拳 \\ \hline
	\bfseries 8 & ?--2021.5.20 & 复盘实际实现,赋能毕设论文 \\ \hline
    \bfseries 9 & 2021.5.21--2021.5.31 & 论文评审及查重 \\ \hline
	\bfseries 10 & 2021.6.1--2021.6.6 & 答辩报告会 \\ \hline
    \end{tabular}
\end{table}

% 列出参考文献
\section{主要参考文献}
% 修正目录超链接问题
\phantomsection{}
\printbibliography[heading=none]

% 生成小组评审意见表
\makeassessment
\end{document}
